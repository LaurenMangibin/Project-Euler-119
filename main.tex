\documentclass{article}
\usepackage[margin = 1 in]{geometry}
\usepackage[parfill]{parskip}
\input{Preamble.tex}


\begin{document}

\begin{flushright}
Lauren Mangibin
\end{flushright}

\begin{center}
    \underline{\textbf{Project Euler \#119}}
\end{center}

Problem:

The number 512 is interesting because it is equal to the sum of its digits raised to some power: 5+1+2 = 8, and $8^3$ = 512. Another example of a number with this property is 614656 = $28^4$.

We shall define $a_n$ to be the $n$th term of this sequence and insist that a number must contain at least two digits to have a sum.

You are given that $a_2$ = 512 and $a_{10}$ = 614656.

Find $a_{30}$.

Answer and Explanation:
248155780267521\\

\hspace{5mm} This series includes numbers whose digits add to a number that, when raised a power, equal the original number. To create this series, I would first need to raise a set of bases to a set of exponents to produce a list of possibilities for the series. Next, I would need to filter out the numbers that did not conform to this rule. Afterwards, I would need to sort the numbers in least to greatest to find the $n$th term\\

\hspace{5mm} $a_{30}$ would need to be a considerably large number, so I chose the range of 2 to 100 for the base and 2 to 10 for the exponent, giving me a possibility of numbers from 4 to $100^{10}$. If needed, I could increase the range of both the base and/or the exponents. \\

Through the program, I was able to create this list, filter it out, and choose the 30th number.



Code:
    \begin{verbatim}
def digitSum(n):                         #command takes the number and returns 
return sum(map(int, str(n)))             #the sum of the digits


a = []                                   #creates a list
n = 30                                   #number in list wanted
for base in range(2, 100):               #bases between 2-100
    for exponent in range(2, 10):        #exponents between 2-10
        product = base ** exponent 
        if digitSum(product) == base:    #filtering out numbers
            a.append(p)
a.sort()                                 #sorting from least to greatest

print (a[n-1])

  \end{verbatim}
\end{document}
